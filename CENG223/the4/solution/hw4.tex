\documentclass[11pt]{article}
\usepackage[utf8]{inputenc}
\usepackage{float}
\usepackage{amsmath}
\usepackage{amssymb}

\usepackage[hmargin=3cm,vmargin=6.0cm]{geometry}
%\topmargin=0cm
\topmargin=-2cm
\addtolength{\textheight}{6.5cm}
\addtolength{\textwidth}{2.0cm}
%\setlength{\leftmargin}{-5cm}
\setlength{\oddsidemargin}{0.0cm}
\setlength{\evensidemargin}{0.0cm}

% symbol commands for the curious
\newcommand{\setZp}{\mathbb{Z}^+}
\newcommand{\setR}{\mathbb{R}}
\newcommand{\calT}{\mathcal{T}}

\begin{document}

\section*{Student Information } 
%Write your full name and id number between the colon and newline
%Put one empty space character after colon and before newline
Full Name : Ersel Hengirmen  \\
Id Number : 2468015 \\

% Write your answers below the section tags
\section*{Answer 1}
We will first choose the planets and the star then order them.\\
First we have 10 stars and we need to pick only one so 10 comes from there.\\
we have 20 habitable planets and we need to choose 2 from them with: $\binom{20}{2}$\\
we have 80 non-habitable planets and we need to choose 8 from them with: $\binom{80}{8}$\\
now we have chosen our planets and we can choose:\\
10*$\binom{20}{2}*\binom{80}{8}$ different combinations\\
\\
Now before we start ordering them by planet types before ordering them by individual planets\\
lets calculate how many ways we can order habitable planets:\\
Important Note:\textbf{My indexes will be from 0 to 10. O will always show the star.}\\
We will order them by distance to our star.\\
The question wants us that there must be at least 6 non-habitable planets between habitable ones\\
that means there are at least 6 at most 8 (1+8+1=10) lets calculate each case:\\
\\
if there are 6 spaces between them habitable planets will be at 3 different positions:\\
1 8\\
2 9\\
3 10\\
if there are 7 spaces between them habitable planets will be at 2 different positions:\\
1 9\\
2 10\\
if there are 8 spaces between them habitable planets will be at 1 different positions:\\
1 10\\
so there are $(3+2+1)=6$ ways for planets to be ordered by type\\
and we can simply calculate their in-group order by 8! and 2!:\\
so at the end we have:\\
10*$\binom{20}{2}*\binom{80}{8}*6*8!*2!=$\\
$10*(20*19)/2!*(80*79*78*77*76*75*74*73)/8!*6*8!*2!=$\\
$10*(20*19)*(80*79*78*77*76*75*74*73)*6=\textbf{26648126951846400000}$\\
\\
\\
\\
\textbf{Second Solution}:\\
from top part we know that there are 6 different ways to order non-habitable and habitable planets.\\
we have 10 stars so we will just use permutation:\\
1. non-habitable:80 choices\\
2. non-habitable:79 choices\\
3. non-habitable:78 choices\\
4. non-habitable:77 choices\\
5. non-habitable:76 choices\\
6. non-habitable:75 choices\\
7. non-habitable:74 choices\\
8. non-habitable:73 choices\\
1. habitable:20 choices\\
2. habitable:19 choices\\
$10*(20*19)*(80*79*78*77*76*75*74*73)*6=\textbf{26648126951846400000}$\\

\section*{Answer 2}
$a_n=a_n^{h}+a_n^{p}$\\
\\
$a_n = 2a_{n-1} + 15a_{n-2} - 36a_{n-3} + 2^{n}$\\
$a_n - 2a_{n-1} - 15a_{n-2} + 36a_{n-3}=2^{n}$  \\
lets first find homogenous solution:\\
$a_n - 2a_{n-1} - 15a_{n-2} + 36a_{n-3}=0$    homogenous form\\
$r^{3}-2r^{2}-15r-36=0$\\
$(r-3)^{2}*(r+4)=0$\\
$r_{1,2}=3$\\
$r_{3}=-4$\\
$a_n^{h}=(A*n+B)*3^{n}+C*(-4)^n$\\

lets now find the particular solution:\\
since solution is of the form $2^{n}$ we will choose particular solution of the form $a_n^{p}=D*2^{n}$\\
$D*2^{n}-2D*2^{n-1}-15D*2^{n-2}+36D*2^{n-3}=2^{n}$\\
$D*2^{n-3}(8-8-30+36)=2^{n}$\\
$D*(6)=8$\\
$D=4/3$\\
so\\
$a_n^{p}=2^{n+2}/3$\\
\\
since $a_n=a_n^{h}+a_n^{p}$\\
$a_n=(A*n+B)*3^{n}+C*(-4)^n+2^{n+2}/3$\\







\section*{Answer 3}
Well first of all we need to find our base $a_1$:\\
since there must be odd number of odd digits if this string has\\
only one number then it must be odd and since there are only 5 odd numbers\\
$a_1=5$\\
now lets find recurrence relation:\\
now to create an n length code we can do 2 things:\\
\\
1-Take a before generated password of length n-1 and add an even number to it\\
so that our number of odd numbers will remain odd.\\
In this case we will have $a_{n-1}*5$ passwords that has n length.
\\
2-Now take a n-1 lengthed digit string thats not an acceptable password\\
since its not acceptable it will have even number of odd numbers\\
which means we will need to add an odd number at the end of it to create a password\\
so since the total number of digit strings that has n-1 length is $10^{n-1}$\\
and $a_{n-1}$ of them have odd number of odd digits.\\
For this case we will have $(10^{n-1}-a_{n-1})*5$ passwords that has n length\\
\\
$a_{n}=a_{n-1}*5+(10^{n-1}-a_{n-1})*5=10^{n-1}*5$\\
$a_{n}=10^{n-1}*5$\\
$a_{n-1}=10^{n-2}*5$\\
$10*a_{n-1}=10^{n-1}*5$\\
So our recurrance relation is:\\
$a_{n}=10*a_{n-1}$\\
With initial condition: $a_{0}=5$

\section*{Answer 4}
 Our recurrence relation says: $a_{k} = 3a_{k-1} -3a_{k-2} +a_{k-3}$\\
$a_0=1$\\
$a_1=3$\\
$a_2=6$\\
above conditions are our premises\\
First of all let $F(x)=\sum_{n=0}^{\infty}a_{n}x^{n}$\\
since k-3 is there we can show our recurrence relation as:
$a_{n} = 3a_{n-1} -3a_{n-2} +a_{n-3}$ for n$\geq$3\\
now we will replace above formula with below one\\
\\
$\sum_{n=3}^{\infty}a_{n}x^{n}=3\sum_{n=3}^{\infty}a_{n-1}x^{n}-3\sum_{n=3}^{\infty}a_{n-2}x^{n}+\sum_{n=3}^{\infty}a_{n-3}x^{n}$\\
\\

Now we will first create left hand side of this equation:\\
since summation starts from n=3 it does not include 0,1 and 2\\
Because of that left hand side of the equation is:\\
$F(x)-a_{0}-a_{1}x-a_{2}x^{2}=F(x)-1-3x-6x^{2}$\\

Now we will create right hand side of this equation:\\
\\
$=3x\sum_{n=3}^{\infty}a_{n-1}x^{n-1}-3x^{2}\sum_{n=3}^{\infty}a_{n-2}x^{n-2}+x^{3}\sum_{n=3}^{\infty}a_{n-3}x^{n-3}$\\
\\
$\sum_{n=3}^{\infty}a_{n-1}x^{n-1}$ has elements from 2 to infinity and does not contain $a_{0}$ and $a_{1}$\\
$\sum_{n=3}^{\infty}a_{n-2}x^{n-2}$ has elements from 1 to infinity and does not contain $a_{0}$\\
$\sum_{n=3}^{\infty}a_{n-3}x^{n-3}$ has elements from 0 to infinity\\
because of that rhs of the equation is equal to:\\
\\
$=3x(F(x)-a_{0}-a_{1}x)-3x^{2}(F(x)-a_{0})+x^{3}F(x)$\\
$=3x(F(x)-1-3x)-3x^{2}(F(x)-1)+x^{3}F(x)$\\
$=F(x)(x^{3}-3x^{2}+3x)-6x^{2}-3x$\\
\\
$=F(x)(x^{3}-3x^{2}+3x)-6x^{2}-3x$\\
lets merge rhs with lhs:\\
$F(x)-1-3x-6x^{2}=F(x)(x^{3}-3x^{2}+3x)-6x^{2}-3x$\\
$F(x)(x^{3}-3x^{2}+3x-1)=-1$\\
$F(x)=-1/(x^{3}-3x^{2}+3x-1)$\\
$F(x)=-1/(x-1)^3$\\
$F(x)=1/(1-x)^3$\\
\\
We will use:
$\sum_{n=0}^{\infty}x^{n}=1/(1-x)\leftrightarrow$(1,1,1,1,1,...) lets take its derivative 2 times to create something similar to what we need\\

$1/(1-x)^{2}\leftrightarrow$(1,2,3,4,...,n+1,...)    first derivative\\
$2/(1-x)^{3}\leftrightarrow$(2,6,12,20,...,(n+1)*(n+2),...)    second derivative\\
and after this by dividing to 2 we get:\\
$F(x)=1/(1-x)^{3}\leftrightarrow$(1,3,6,10,...,(n+1)*(n+2)/2,...)
which means:\\
$a_n=(n+1)*(n+2)/2$\\




\section*{Answer 5}
\paragraph{a.}
By its definition R is an equavilance relation if it is transitive, symmetric and reflexive\\
The question tels us that ((a, b),(c, d))$\in$R if and
only if a+d=b+c\\
\\
1-lets see if it is reflexive:\\
for it to be reflexive ((a,b),(a,b)) must be in R\\
since a+b=b+a it is reflexive\\
\\
2-lets see if it is symmetric:\\
By the definition of symmetricity if ((a, b),(c, d))$\in$R than ((c, d),(a, b))$\in$R\\
lets assume ((a, b),(c, d))$\in$R that means:a+d=b+c\\
and since a+d=b+c holds, c+b=a+d must also hold(since they are the same thing)\\
which means:\\
(((a, b),(c, d))$\in$R)$\rightarrow$(((c, d),(a, b))$\in$R)\\
and because of that R is symmetric\\
\\
3-lets see if it is transitive:\\
By the definition of transitive: 
if ((a, b),(c, d))$\in$R and ((c, d),(e, f))$\in$R\\
than ((a, b),(e, f))$\in$R must hold\\
Let ((a, b),(c, d)),((c, d),(e, f))$\in$R\\
that means a+d=b+c and c+f=d+e\\
now we need to probe a+f=b+e\\
we know that a+d=b+c and c+f=d+e so:\\
(a+d)+(c+f)=(b+c)+(d+e)\\
a+d+c+f=b+c+d+e now lets extract c+d from both sides\\
a+d+c+f-(c+d)=b+c+d+e-(c+d)\\
a+f=b+e\\
which means:\\
((((a, b),(c, d))$\in$R)$\land$(((a, b),(c, d))$\in$R))$\rightarrow$(((a, b),(e, f))$\in$R)\\

and because of that R is transitive\\
\\
Since R is transitive, symmetric and reflexive R is an equivalence relation

\paragraph{b.}
By the description of R, ((a,b),(1,2))$\in$R is only possible if a+2=b+1 which means b=a+1\\
That means Equavilance class of (1,2) with respect to R is: \{(a,a+1):$\in Z^{+}$\}\\

\end{document}
