\documentclass[12pt]{article}
\usepackage[utf8]{inputenc}
\usepackage[dvips]{graphicx}
\usepackage{epsfig}
\usepackage{fancybox}
\usepackage{verbatim}
\usepackage{array}
\usepackage{latexsym}
\usepackage{alltt}
\usepackage{float}
\usepackage{amsmath}
\usepackage{hyperref}
\usepackage{listings}
\usepackage{color}
\usepackage[hmargin=3cm,vmargin=5.0cm]{geometry}
\topmargin=-1.8cm
\addtolength{\textheight}{6.5cm}
\addtolength{\textwidth}{2.0cm}
\setlength{\oddsidemargin}{0.0cm}
\setlength{\evensidemargin}{0.0cm}

\newcommand{\HRule}{\rule{\linewidth}{1mm}}
\newcommand{\kutu}[2]{\framebox[#1mm]{\rule[-2mm]{0mm}{#2mm}}}
\newcommand{\gap}{ \\[1mm] }

\newcommand{\Q}{\raisebox{1.7pt}{$\scriptstyle\bigcirc$}}

\lstset{
    %backgroundcolor=\color{lbcolor},
    tabsize=2,
    language=C++,
    basicstyle=\footnotesize,
    numberstyle=\footnotesize,
    aboveskip={0.0\baselineskip},
    belowskip={0.0\baselineskip},
    columns=fixed,
    showstringspaces=false,
    breaklines=true,
    prebreak=\raisebox{0ex}[0ex][0ex]{\ensuremath{\hookleftarrow}},
    %frame=single,
    showtabs=false,
    showspaces=false,
    showstringspaces=false,
    identifierstyle=\ttfamily,
    keywordstyle=\color[rgb]{0,0,1},
    commentstyle=\color[rgb]{0.133,0.545,0.133},
    stringstyle=\color[rgb]{0.627,0.126,0.941},
}


\begin{document}



\noindent
\HRule \\[3mm]
\small
\begin{tabular}[b]{lp{3.8cm}r}
{} Middle East Technical University &  &
{} Department of Computer Engineering \\
\end{tabular} \\
\begin{center}

                 \LARGE \textbf{CENG 223} \\[4mm]
                 \Large Discrete Computational Structures \\[4mm]
                \normalsize Fall '2020-2021 \\
                    \Large Homework 3 \\
                \normalsize Student Name and Surname: Ersel Hengirmen  \\
                \normalsize Student Number: 2468015\\
\end{center}
\HRule


\section*{Question 1}
$(2^{22} + 4^{44} + 6^{66} + 8^{80} + 10^{110}) $mod $11 \equiv ?$\\
Since 11 is prime and does not divide 2,4,6,8 and 10, by fermat's little theorem:\\
$(2^{11-1})$mod $1\equiv 1$\\
$(4^{11-1})$mod $11\equiv 1$\\
$(6^{11-1})$mod $11\equiv 1$\\
$(8^{11-1})$mod $11\equiv 1$\\
$(10^{11-1})$mod $11\equiv 1$\\

$(2^{22} + 4^{44} + 6^{66} + 8^{80} + 10^{110}) $mod $11 \\
\equiv (2^{10*2+2} + 4^{10*4+4} + 6^{10*6+6} + 8^{10*8} + 10^{10*11}) $mod $11 $\\
$\equiv (1^{2}*2^{2}+1^{4}*4^{4}+1^{6}*6^{6}+1^{8}+1^{11})$mod $11$\\
$\equiv (4+256+46656+1+1)$mod $11\equiv (4+3+5+1+1)$mod $11\equiv (14)$mod $11$\\
$\equiv 3$\\

\section*{Question 2}

\begin{center}
\begin{tabular}{ c c c }
& \textbf{gcd(5n+3,7n+4)} &\\ \cline{2-2}
\textbf{Higher} &\textbf{Lower} &\textbf{Combination} \\ \cline{1-3}
7n+4&    5n+3  &      7n+4=1*(5n+3)+(2n+1)\\
5n+3 &   2n+1 &       5n+3=2*(2n+1)+(n+1)\\
2n+1  &  n+1 &        2n+1=1*(n+1)+n\\
n+1    & n  &         n+1=1*(n)+1\\
n  &     1 &          n=n*(1)\\
1   &    0&
\end{tabular}
\end{center}

By Euclid's algorithm gcd(5n + 3, 7n + 4)=1\\

\section*{Question 3}

$m^{2}=n^{2}+kx$\\
For this problem we know that m,n and k are integer numbers and x is a prime number.\\
$m^{2}=n^{2}+kx$\\
\\
Now we will extract $n^{2}$ from both sides\\
$m^{2}-n^{2}=kx$\\
\\
by the definition of divisibility $x|m^{2}-n^{2}$ is clear but to make it more clear\\
since $m^{2}-n^{2}=(m+n)*(m-n)$ we can change the above equation to:\\
$(m+n)*(m-n)=kx$\\
\\
Now we will divide every side by x:\\
$\dfrac{(m+n)*(m-n)}{x}=k$\\
\\
We know that m and n are integers which means m+n and m-n must also be integers. \\
we can easily see that $x|m^{2}-n^{2}$ since k is also an integer\\
since x is prime it can not be written as multiplication of 2 prime or composite numbers\\
because of that either m+n or m-n must be divisible to x(proof of this is below). Otherwise k wouldnt be and integer.\\

Proof:\\
Firstly dont forget x is prime.\\
We will prove that if x$|$((m+n)(m-n)) than $(x|(m+n))\lor(x|(m-n))$\\ 
suppose that gcd(x,(m+n))=1 (if its not then x divides (m+n) since x is prime and we proved $((x|(m+n))\lor(x|(m-n)))$\\
By Bezout's Lemma there are integers a,b such that xa+(m+n)b=1\\
xa+(m+n)b=1\\
(m-n)xa+(m+n)(m-n)b=(m-n)\\
(m-n)xa has x so it is divisible by x\\
(m+n)(m-n)b has (m+n)(m-n) so it is divisible by x\\
(m-n)=(m-n)xa+(m+n)(m-n)b both of them are divisible by x so (m-n) is divisible by x\\
and my proof is complete\\

\section*{Question 4}

I will show that for all n such that n $\geq$ 1 the following is true:\\
$1 + 4 + 7 + ...+ (3n-2) = \dfrac{n(3n-1)}{2}$\\
Note: I will use P(k) as the validity of above equality for an arbitrary integer k$\geq$1
\\
\textbf{Base Step}:\\
for n =1\\
1=1*(3-1)/2=1\\
so it is valid for our base case n=1\\
\\
\textbf{Inductive Step}: Assume that P(k) is valid for an arbitrary integer k$\leq$1 \\
now we will prove 1+4+7...+(3k-2)+(3(k+1)-2)=$\dfrac{(k+1)(3(k+1)-1)}{2}$ is valid (P(k+1))i:\\
\\
1+4+7...+(3k-2)+(3(k+1)-2)\\
=(1+4+7+.....(3k-2))+(3(k+1)-2)\\
Since we assumed that P(k) is valid $(1+4+7+.....(3k-2))=\dfrac{k(3k-1)}{2}$ so using that\\
\\
$=\dfrac{k(3k-1)}{2}+3k+1$\\
\\
$=\dfrac{k(3k-1)+6k+2}{2}$\\
\\
$=\dfrac{k(3k-1)+(3k-1)+3*(k+1)}{2}$\\
\\
$=\dfrac{(k+1)(3k-1)+3*(k+1)}{2}$\\
\\
$=\dfrac{(k+1)(3k+2)}{2}$\\
$=\dfrac{(k+1)(3(k+1)-1)}{2}$\\
From these we saw that if P(k) is valid P(k+1) must also be valid\\
This completes the inductive step\\
\\
By mathematical induction $1 + 4 + 7 + ...+ (3n-2) = \dfrac{n(3n-1)}{2}$ is true for all integers n$\geq$1


\end{document}

